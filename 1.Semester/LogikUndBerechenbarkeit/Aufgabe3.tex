% !TEX encoding = UTF-8 Unicode
\documentclass[10pt,ngerman]{scrartcl}
\usepackage{url,bm,tikz,a4wide}
\usepackage[utf8]{inputenc}
\usepackage{booktabs}
\usepackage{amsmath,amssymb}
\usepackage[ngerman]{babel}
\usepackage{graphicx,tikzsymbols}
\usepackage{tikz}
\usepackage{float}
\usepackage{graphicx}
\usepackage{arev}

\renewcommand{\theenumi}{\arabic{paragraph}.\alph{enumi}}
\renewcommand{\theenumii}{\roman{enumii}}
\renewcommand{\labelenumi}{\roman{enumi})}

\def\firstcircle{(90:1.75cm) circle (2.5cm)}
\def\secondcircle{(210:1.75cm) circle (2.5cm)}
\def\thirdcircle{(330:1.75cm) circle (2.5cm)}

%SPEZIELLE KOMMENTARE FÜR LOGIK UND BERECHENBARKEIT
\newcommand{\w }{\texttt{ W }}
\newcommand{\f }{\texttt{ F }}

\setcounter{secnumdepth}{-1}


\begin{document}

\begin{figure}[htbp]
\begin{minipage}[b]{0.50\linewidth}
\begin{Large}

%HIER PERSÖNLICHE DATEN EINTRAGEN
	\textbf{Name:}\\
	Gritsch 			\\
	\textbf{Vorname:}\\
	Lukas 				\\
	\textbf{Matrikelnummer:}\\
	2210836017

\end{Large}
\end{minipage}
\begin{minipage}[b]{0.50\linewidth}
\begin{flushright}
\begin{Huge}
%% HIER LEHRVERANSTALTUNG EINGEBEN/EINKOMMENTIEREN
%\textbf{eCollaboration}\\
%\textbf{Logik und \\Berechenbarkeit}\\
%\textbf{Mathematik für\\ Software Engineering}\\
\end{Huge}
\vspace{10px}
\begin{large}
%% HIER SEMESTER EINGEBEN
Wintersemester 2022/23
\end{large}
\end{flushright}
\end{minipage}
\end{figure}

\vspace{20px}
\begin{huge}
\noindent

%HIER NUMMER DES ÜBUNGSZETTELS EINTRAGEN
\textbf{Übungsblatt 3}
\end{huge}

%HIER DIE JEWEILIGEN AUFGABENNUMMERN UND -NAMEN EINTRAGEN
\pagebreak
\section{IstElementVon, Gleichheit und Teilmengen}
Prüfen Sie für folgende Objekte bzw. Mengen ob die angegebenen Beziehungen Richtig
oder Falsch sind. Begründen Sie Ihre Antwort kurz
\subsection{1.a)}
Die Bezeichnung $\{\spadesuit,\diamondsuit\} \notin \underbrace{\{\{\spadesuit,\varheart\},\clubsuit,\diamondsuit\}}_{B}$ ist \textbf{Wahr}. In der Menge B sind die Elemente $\{\spadesuit,\varheart\},\clubsuit$ und $\diamondsuit$ enthalte. Da aber Das Element $\{\spadesuit,\diamondsuit\}$ befindet sich aber nicht darin

\subsection{1.b)}
Die Bezeichnung $42 \in \mathbb{N}$ ist \textbf{Wahr}. Die Menge $\mathbb{N}$ beinhaltet die positiven ganzen Zahlen von 0 bzw. 1 (kommt auf die Definiton an) bis $\infty$. Dort kommt auch das Element $42$ vor.

\subsection{1.c)}
%$\heartsuit\varheart\diamondsuit\vardiamond\clubsuit\spadesuit$
Die Bezeichnung $\diamondsuit,\clubsuit \in \underbrace{\{\{\diamondsuit\},\{\clubsuit\}\}}_{B}$ ist \textbf{Falsch}. Die Menge B beinhaltet die Elemente $\{\diamondsuit\}$ und $\{\clubsuit\}$. Da das Element $\diamondsuit$ nicht das Gleiche ist wie die Menge $\{\diamondsuit\}$, es gilt also $\diamondsuit \neq \{\diamondsuit\}$. Sind auch die Elemente $\diamondsuit$ und $\clubsuit$ nicht in der Menge B vorhanden.

\subsection{1.d)}
Die Behauptung $\underbrace{\{Kiwi,Apfel,Birne,Orange,Kiwi\}}_{A} = \underbrace{\{Apfel,Birne,Orange,Kiwi\}}_{B}$ ist \textbf{Wahr}. Die Gleichheit setzt voraus, dass jedes Element welches in der Menge A enthalten ist auch in der Menge B enthalten sein muss und jedes Element welches in der Menge B Enthalten ist auch in der Menge A Enthalten sein muss. Es gilt also $A=B: \Leftrightarrow \forall x: x \in A \Leftrightarrow x\in B$. Alle Elemente der Menge A sind in der Menge B Enthalten und alle Elemente der Menge B sind in der Menge A Enthalten, deshalb ist diese Behauptung Wahr.

\subsection{1.e)}
Die Behauptung $b \in \underbrace{\{a,b,c\}}_{B}$ ist \textbf{Wahr}. Die Menge B beinhaltet die Elemente $a,b$ und $c$. Da $b$  darin Enthalten ist ist diese Behauptung wahr.

\subsection{1.f)}
Die Behauptung $\underbrace{\{\delta,\epsilon\,\delta\}}_{A} = \underbrace{\{\epsilon,\delta,\epsilon\}}_{B}$ ist \textbf{Wahr}. Jedes Element aus A ist in B Enthalten und jedes Element aus B ist in A Enthalten. Deshalb ist die Behauptung wahr.

\subsection{1.g)}
Die Behauptung $\underbrace{\{Apfel,Birne,Orange\}}_{A} \in \underbrace{\{Obst\}}_{B}$ ist \textbf{Falsch}.
Die Menge B beinhaltet das Elemente $Obst$ da in dieser Behauptung aber gesagt wird, dass die Menge A ein Element der Menge B ist, ist dies Falsch. Also gilt $\underbrace{\{Apfel,Birne,Orange\}}_{A} \notin \underbrace{\{Obst\}}_{B}$

\subsection{1.h)}
Die Behauptung $\underbrace{\{Apfel,Birne,Orange\}}_{A} = \underbrace{\{Apfel,Birne,Apfel,Orange,Banane\}}_{B}$ ist \textbf{Falsch}. Es sind zwar alle Elemente der Menge A in B Enthalten, aber nicht alle Elemente der Menge B in A. Das Element $Banane$ ist nur in der Menge B enthalten.

\section{Mengenbildung durch Aussage}
\subsection{2.a)}
Mengenbildung durch Aussage:
\begin{align*}
M:=\{x^2 | x \in \mathbb{Z} \wedge (-10 \leq x \leq 10) \wedge Gerade(x)\}
\end{align*}
Expliziete Mengenbildung:
\begin{align*}
M:=\{0,4,16,36,64,100\}
\end{align*}
\subsection{2.b)}
Mengenbildung durch Aussage:
\begin{align*}
M:=\{x^2+2x|(2\leq x \leq 8) \wedge Gerade(x)\}
\end{align*}
Expliziete Mengenbildung:
\begin{align*}
 M:=\{8,24,48,80\}
\end{align*}
\subsection{2.c)}
Mengenbildung durch Aussage:
\begin{align*}
M:=\{x|x \in \mathbb{N} \wedge x \leq 8\}
\end{align*}
Expliziete Mengenbildung:
\setcounter{equation}{0}
\begin{align}
M:=\{0,1,2,3,4,5,6,7,8\} && \text{Wenn man } \mathbb{N} \text{ mit 0 definiert} \\
M:=\{1,2,3,4,5,6,7,8\} && \text{Wenn man } \mathbb{N} \text{ ohne 0 definiert}
\end{align}
\subsection{2.d)}
Mengenbildung durch Aussage:
\begin{align*}
M:=\{x^2|x \leq 20 \wedge Prim(x) \wedge Gerade(x^2)\}
\end{align*}
Expliziete Mengenbildung:
\setcounter{equation}{0}
\begin{align*}
M:=\{4\}
\end{align*}
\pagebreak
\section{Venn-Diagramme}
\subsection{3.a)}
\begin{center}
\begin{tikzpicture}
\begin{scope}
\fill[orange] \firstcircle;
\end{scope}
\begin{scope}
\fill[orange] \secondcircle;
\end{scope}
\begin{scope}
\fill[orange] \thirdcircle;
\end{scope}
\begin{scope}
\clip \secondcircle;
\fill[white] \firstcircle;
\end{scope}
\begin{scope}
\clip \thirdcircle;
\fill[white] \secondcircle;
\end{scope}
\begin{scope}
\clip \firstcircle;
\fill[white] \thirdcircle;
\end{scope}
\begin{scope}
\clip \firstcircle;
\clip \secondcircle;
\fill[orange] \thirdcircle;
\end{scope}
\draw \firstcircle node[text=black,above] {$A$};
\draw \secondcircle node [text=black,below left] {$B$};
\draw \thirdcircle node [text=black,below right] {$C$};
\end{tikzpicture}
\end{center}
\setcounter{equation}{0}
\begin{align*}
M = (A\textbackslash B \textbackslash C)\cup(B\textbackslash C \textbackslash A)\cup(C \textbackslash B \textbackslash A)\cup(A\cap B \cap C)
\end{align*}
\subsection{3.b)}
\begin{center}
\begin{tikzpicture}
\begin{scope}
\clip \secondcircle;
\fill[orange] \firstcircle;
\end{scope}
\begin{scope}
\clip \thirdcircle;
\fill[orange] \secondcircle;
\end{scope}
\begin{scope}
\clip \firstcircle;
\fill[orange] \thirdcircle;
\end{scope}
\draw \firstcircle node[text=black,above] {$A$};
\draw \secondcircle node [text=black,below left] {$B$};
\draw \thirdcircle node [text=black,below right] {$C$};
\end{tikzpicture}
\end{center}
\setcounter{equation}{0}
\begin{align*}
M = (A\cap B) \cup (A\cap C) \cup (A \cap C)
\end{align*}
\pagebreak
\subsection{3.c)}
\begin{center}
\begin{tikzpicture}
\begin{scope}
\fill[orange] \firstcircle;
\end{scope}
\begin{scope}
\fill[orange] \thirdcircle;
\end{scope}
\begin{scope}
\clip \secondcircle;
\fill[white] \firstcircle;
\end{scope}
\draw \firstcircle node[text=black,above] {$A$};
\draw \secondcircle node [text=black,below left] {$B$};
\draw \thirdcircle node [text=black,below right] {$C$};
\end{tikzpicture}
\end{center}
\begin{align*}
 M = (A \textbackslash B) \cup (C \textbackslash A)
\end{align*}

\section{Große Vereinigung und großer Durchschnitt}
\setcounter{equation}{0}
\begin{align*}
A:= \{\{Apfel,Birne\},\{Birne,Banane,Kiwi\},\{Birne,Kiwi\},\{Kiwi,Birne\}\}\\
B:= \{\{a\},\{\varheart,\spadesuit\}\{b,c\}\}
\end{align*}

\subsection{4.a)}
\setcounter{equation}{0}
\begin{align*}
\bigcup A =\{Apfel,Birne,Banane,Kiwi\}
\end{align*}

\subsection{4.b)}
\setcounter{equation}{0}
\begin{align*}
\bigcap A =\{Birne\}
\end{align*}

\subsection{4.c)}
\setcounter{equation}{0}
\begin{align*}
\bigcup B =\{a,\varheart,\spadesuit,b,c\}
\end{align*}

\subsection{4.d)}
\setcounter{equation}{0}
\begin{align*}
\bigcap B =\{\}
\end{align*}
\pagebreak
\section{Potenzmengen und Mächtigkeit}
\subsection{5.a)}
\setcounter{equation}{0}
\begin{align*}
Pot(\{Apfel,Banane,Kiwi\}) = \{\{\},\{Apfel\},\{Banane\},\{Kiwi\},\{Apfel,Banane\},\\
\{Apfel,Kiwi\},\{Banane,Kiwi\},\{Apfel,Banane,Kiwi\}\}
\end{align*}
\subsection{5.b)}
\setcounter{equation}{0}
\begin{align}
Pot(\{Pot(\{Birne\})\}) = Pot(\{\{\},Birne\})\\
Pot(\{\{\},Birne\}) = \{\{\},\{\{\}\},\{Birne\},\{\{\},Birne\}\}
\end{align}
\subsection{5.c)}
\setcounter{equation}{0}
\begin{align*}
|\{Apfel, Birne, Banane, Kiwi, Orange, Mango\}| = 6
\end{align*}

\subsection{5.d)}
\setcounter{equation}{0}
\begin{align*}
|Pot(\{a, b, c, d, \{a, b, c\}, \{d, e, f , g\}\})| = 2^6 = 64
\end{align*}

\section{Tupel und Kartesisches Produkt}
\setcounter{equation}{0}
\begin{align*}
D = (Apfel, Birne, Banane), I = (Banane, Apfel, Birne), B = (\alpha, \beta, \gamma)\\
S = (\alpha, \beta, \gamma), E = \{Apfel, Birne\}, M = \{\alpha, \beta, \{\gamma, \delta\}\}, C = \{\}
\end{align*}

\subsection{6.a)}
Die Behauptung $D = I$ ist \textbf{Falsch}. D und I sind Tupel, damit diese gleiche sind muss auch die Reihenfolge der Elemente stimmen.

\subsection{6.b)}
Die Behauptung $B = S$ ist \textbf{Wahr}

\subsection{6.c)}
\setcounter{equation}{0}
\begin{align*}
E x D = \{(Apfel,Apfel),(Apfel,Birne),(Apfel,Banane),\\(Birne,Apfel),(Birne,Birne),(Birne,Banane)\}
\end{align*}

\subsection{6.d)}
\setcounter{equation}{0}
\begin{align*}
E x M = \{(Apfel,\alpha),(Apfel,\beta),(Apfel,\{\gamma, \delta\}),\\(Birne,\alpha),(Birne,\beta),(Birne,\{\gamma, \delta\})\}
\end{align*}
\end{document}
