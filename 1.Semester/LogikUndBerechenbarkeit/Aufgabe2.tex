% !TEX encoding = UTF-8 Unicode
\documentclass[10pt,ngerman]{scrartcl}
\usepackage{url,bm,tikz,a4wide}
\usepackage[utf8]{inputenc}
\usepackage{booktabs}
\usepackage{amsmath,amssymb}
\usepackage[ngerman]{babel}
\usepackage{graphicx,tikzsymbols}
\usepackage{tikz}
\usepackage{float}
\usepackage{graphicx}

\renewcommand{\theenumi}{\arabic{paragraph}.\alph{enumi}}
\renewcommand{\theenumii}{\roman{enumii}}
\renewcommand{\labelenumi}{\roman{enumi})}

%SPEZIELLE KOMMENTARE FÜR LOGIK UND BERECHENBARKEIT
\newcommand{\w }{\texttt{ W }}
\newcommand{\f }{\texttt{ F }}

\setcounter{secnumdepth}{-1}


\begin{document}

\begin{figure}[htbp]
\begin{minipage}[b]{0.50\linewidth}
\begin{Large}

%HIER PERSÖNLICHE DATEN EINTRAGEN
	\textbf{Name:}\\
	Gritsch 			\\
	\textbf{Vorname:}\\
	Lukas 				\\
	\textbf{Matrikelnummer:}\\
	2210836017

\end{Large}
\end{minipage}
\begin{minipage}[b]{0.50\linewidth}
\begin{flushright}
\begin{Huge}
%% HIER LEHRVERANSTALTUNG EINGEBEN/EINKOMMENTIEREN
%\textbf{eCollaboration}\\
%\textbf{Logik und \\Berechenbarkeit}\\
%\textbf{Mathematik für\\ Software Engineering}\\
\end{Huge}
\vspace{10px}
\begin{large}
%% HIER SEMESTER EINGEBEN
Wintersemester 2022/23
\end{large}
\end{flushright}
\end{minipage}
\end{figure}

\vspace{20px}
\begin{huge}
\noindent

%HIER NUMMER DES ÜBUNGSZETTELS EINTRAGEN
\textbf{Übungsblatt 2}
\end{huge}

%HIER DIE JEWEILIGEN AUFGABENNUMMERN UND -NAMEN EINTRAGEN
\pagebreak
\section{All-und Existenzaussagen}
\subsection{1.a)}
Negieren der Aussage $\forall x \forall y \neg \exists z : ((x*y) > 2z)$:
\setcounter{equation}{0}
\begin{align}
\neg(\forall x \forall y \neg \exists z : ((x*y) > 2z))\\
\exists x \neg(\forall y \neg \exists z : ((x*y) > 2z))\\
\exists x \exists y \neg (\neg \exists z : ((x*y) > 2z))\\
\exists x \exists y \exists z : \neg ((x*y) > 2z)\\
\exists x \exists y \exists z : (\neg (x*y) \leq 2z)
\end{align}
\subsection{1.b)}
Negieren der Aussage $\forall x \neg \exists y : (x^2 \geq y)$:
\setcounter{equation}{0}
\begin{align}
\neg(\forall x \neg \exists y : (x^2 \geq y))\\
\exists x \neg(\neg \exists y : (x^2 \geq y)\\
\exists x \exists y : \neg (x^2 \geq y)\\
\exists x \exists y : x^2 < y
\end{align}
\subsection{1.c)}
Negieren der Aussage $\forall x : \text{isPrime(}x\text{)}$:
\setcounter{equation}{0}
\begin{align}
\neg(\forall x : \text{isPrime(}x\text{)})\\
\exists x :\neg(\text{isPrime(}x\text{)})
\end{align}
\pagebreak
\section{Prädikatendefinition 1}
Finden Sie den bzw. die Fehler in den folgenden Formalisierungen des folgenden deut-
schen Satzes:
\setcounter{equation}{0}
\begin{center}
\text{„Es existiert für jede ungerade natürliche Zahl unter Zehn eine ungerade natürliche Zahl,}\\\text{
die dadurch entsteht, dass man die erste Zahl quadriert“}
\end{center}
\subsection{2.a}
\setcounter{equation}{0}
\begin{align*}
\forall x \exists y: \text{Ungerade(}x < 10\text{)}^2 = \text{Ungerade(}y\text{)}
\end{align*}
Ungerade() überprüft, ob ein Zahl oder Variable ungerade ist. $x<10$ ist aber eine atomare Aussage und kann nicht auf ungerade geprüft werden. Da Ungerade() ein Prädikat ist nimmt dieses den Zustand\w oder\f ein. Dies zu quadrieren gibt keinen Sinn. Des Weiteren fehlt hier der Zusammenhang zwischen $x$ und $y$ also, dass $y = x^2$ ist.
\subsection{2.b}
\setcounter{equation}{0}
\begin{align*}
\forall x \exists y: \text{Ungerade(}y\text{)} = (x^2) \wedge \text{Ungerade(}x\text{)} \wedge (x < 10)
\end{align*}
In dieser Aussage fehlt der Zusammenhang zwischen $x$ und $y$ also, dass $y = x^2$ ist. $x^2$ ist eine Funktion und gibt eine Zahl als Rückgabewert retour. Eine Zahl mit Wahrheitswerten zu \glqq verunden\grqq{}  ist Falsch bzw. gibt keinen Sinn.
\subsection{2.c}
\setcounter{equation}{0}
\begin{align*}
\forall x \exists y: (x < 10) \wedge \text{Ungerade(}x\text{)} \color{blue}\wedge\color{black} (y = x^2) \wedge \text{Ungerade(}y\text{)}
\end{align*}
In dieser Aussage würde ich das markierte und durch ein Implikation ($\Rightarrow$) auswechseln. Ansonsten ist diese Aussage meiner Meinung nach korrekt.
\pagebreak
\section{Prädikatendefinition 2}
\subsection{3.a}
Die Aussage $P(x, y)$ ist genau dann wahr, wenn $x$ oder $y$ eine Primzahl ist und das
Produkt von beiden eine ungerade Zahl ergibt.
\setcounter{equation}{0}
\begin{align}
\text{Ungerade}(x):\Leftrightarrow \exists y : x = 2y +1\\
\text{Prim}(x): \Leftrightarrow x \ne 1 \wedge \neg (\exists y : y \ne 1 \wedge y \ne x \wedge y|x)\\
P(x,y) :\Leftrightarrow (\text{Prime}(x) \vee \text{Prime}(y)) \Rightarrow \text{Ungerade}(x * y)
\end{align}
\subsection{3.b}
Die Aussage $T(x)$ ist genau dann wahr, wenn x das Produkt aus den Quadraten zweier
einstelliger Zahlen (im Dezimalsystem) ist.
\setcounter{equation}{0}
\begin{align*}
T(x) :\Leftrightarrow ((1 \leq z_1 \wedge z_1 \geq 9) \wedge (1 \leq z_2 \wedge z_2 \geq 9)) \Rightarrow (x = z_1^2*z_2^2)
\end{align*}
\section{Formale Problemspezifikation}
\subsection{4.a}
\textbf{ggT}: Zu zwei gegebenen natürlichen Zahlen soll der grösste gemeinsame Teiler (ggT) gefunden werden.\vspace{0.5 cm}\\
Gegeben: $2 \text{ Zahlen} \in \mathbb{N}$; $x \in \mathbb{N}; y \in \mathbb{N}$\\
Gesucht: $1 \text{ Zahl} \in \mathbb{N}$
\setcounter{equation}{0}
\begin{align*}
\exists x \exists y : (k|x \wedge k|y) \wedge \underset{l \ in \mathbb{N}}{\forall l} : (l|x \wedge l|y) \Rightarrow k \geq l
\end{align*}
\subsection{4.b}
Sie sind zu einer Halloween Party eingeladen, aber kennen die Adresse nicht (Dieser
Satz ist für die Problemspezifikation unerheblich).
Zu einer gegebenen Strasse soll ein Haus gefunden werden, welches einen Kürbis
als Halloween-Dekoration besitzt. Ausserdem soll keines der Nachbarhäuser einen
Vampir als Dekoration besitzen.\vspace{0.5 cm}\\
Gegeben: $ 1 \text{ Straße } y$ \\%\text{,} x \text{ Menge an Häuser in } y \text{ , } z\text{ Menge an Nachbarhäuser zu } x$\\
Gesucht: $ 1 \text{ Haus } x$
\setcounter{equation}{0}
\begin{align}
T := \{\text{Häuser in der Straße }y\}\\
Z := \{\text{Nachbarhäuser von }x \text{ in der Straße }y\}\\
\underset{x \in T}{\exists! x} : \text{hatKürbisAlsDeko}(x) \wedge \underset{n \in Z}{\forall n} : \neg \text{hatVampirAlsDeko}(n)
\end{align}
\pagebreak
\section{Beweise}
\subsection{5.a}
Direkter Beweis für $\text{Gerade}(x) \Rightarrow \text{Gerade}(x^2)$. Bei einem direkten Beweis geht man von einer Aussage $A \Rightarrow B$ aus.  und versucht diesen zu bestätigen.
\setcounter{equation}{0}
\begin{align}
x \in \mathbb{N} \Rightarrow x^2 \in \mathbb{N} && \text{Prämisse} 1\\
\text{Gerade}(x): \Leftrightarrow \exists n \in \mathbb{N} : x = 2n && \text{Prämisse} 2
\end{align}
$x_*$ sei beliebig aber fix und Gerade$(x_*)$ wahr\\
\begin{align}
[\text{Gerade}(x_*) \Rightarrow \exists n \in \mathbb{N} : x_* = 2n && x_* \text{ in Prämisse } 2 \text{ einsetzten}\\
(x_*)^2 = (2n)^2 && x_* \text{quadriert}\\
x_*^2 = 4n^2 \\
x_*^2 = 2 * \underbrace{(2n^2)}_{n^2 \in \mathbb{N} \Rightarrow 2* n^2 \in \mathbb{N}; k := 2n^2}\\
x_*^2 = 2k \Rightarrow \text{Gerade}(x_*^2)] &&\square
\end{align}
\subsection{5.b}
Wiederspruchsbeweis für \glqq Wenn Ungerade$(5x+2)$ ,dann Ungerade$(x)$\grqq{}. Als erstes geht man von einer Aussage $A \Rightarrow \neg B$ aus. Also von $\text{Ungerade}(5x + 2) \Rightarrow \text{Gerade}(x)$:
\setcounter{equation}{0}
\begin{align}
x  \in \mathbb{N} \Rightarrow 5x +2 \in \mathbb{N} && \text{Prämisse} 1 \\
\text{Gerade}(x) : \Leftrightarrow \underset{n \in \mathbb{N}}{\exists n}: x = 2n && \text{Prämisse}2\\
\text{Ungerade}(x) : \Leftrightarrow \underset{e \in \mathbb{N}}{\exists e}: x = 2e +1 && \text{Prämisse}3\\
\underset{k \in \mathbb{N}}{k} := 5x + 2 && \text{Prämisse}4
\end{align}
$x_*$ sei beliebig aber fix und Ungerade$(5x+2)$ wahr\\
\begin{align}
[\text{Gerade}(x_*):\Leftrightarrow \exists n : x_* = 2n && \text{ in Prämisse } 2 \text{ einsetzten}\\
k = 5(2n) +2 = 10 n +2 && x_*\text{ in Prämisse } 4 \text{ einsetzten}\\
\text{Ungerade}(x_*) : \Leftrightarrow \underset{e \in \mathbb{N}}{\exists e}: x_* = 2e +1 && \text{ in Prämisse } 3 \text{ einsetzten}\\
k = 5(2e +1) +2 = 10e +7 && x_*\text{ in Prämisse } 4 \text{ einsetzten}\\
10 e + 7 = 10 n + 2  && k \text{ von Zeile } 6 \text{ und } 8 \text{ gleich setzten}\\
10e+5 = 10n\\
\underbrace{\frac{10 e + 5}{10} = n}_{ e \in \mathbb{N}; n \in \mathbb{N}; \frac{10e +5}{10} \notin \mathbb{N}}] && \qquad \vcenter{\hbox{\includegraphics[width=0.08\linewidth]{blitz}}}
\end{align}
\end{document}
