% !TEX encoding = UTF-8 Unicode
\documentclass[10pt,ngerman]{scrartcl}
\usepackage{url,bm,tikz,a4wide}
\usepackage[utf8]{inputenc}
\usepackage{booktabs}
\usepackage{amsmath,amssymb}
\usepackage[ngerman]{babel}
\usepackage{graphicx,tikzsymbols}
\usepackage{tikz}
\usepackage{float}
\usepackage{graphicx}
\usetikzlibrary{automata, positioning, arrows}


\renewcommand{\theenumi}{\arabic{paragraph}.\alph{enumi}}
\renewcommand{\theenumii}{\roman{enumii}}
\renewcommand{\labelenumi}{\roman{enumi})}

\def\firstcircle{(90:1.75cm) circle (2.5cm)}
\def\secondcircle{(210:1.75cm) circle (2.5cm)}
\def\thirdcircle{(330:1.75cm) circle (2.5cm)}

%SPEZIELLE KOMMENTARE FÜR LOGIK UND BERECHENBARKEIT
\newcommand{\w }{\texttt{ W }}
\newcommand{\f }{\texttt{ F }}

\setcounter{secnumdepth}{-1}


\begin{document}

\begin{figure}[htbp]
\begin{minipage}[b]{0.50\linewidth}
\begin{Large}

%HIER PERSÖNLICHE DATEN EINTRAGEN
	\textbf{Name:}\\
	Gritsch 			\\
	\textbf{Vorname:}\\
	Lukas 				\\
	\textbf{Matrikelnummer:}\\
	2210836017

\end{Large}
\end{minipage}
\begin{minipage}[b]{0.50\linewidth}
\begin{flushright}
\begin{Huge}
%% HIER LEHRVERANSTALTUNG EINGEBEN/EINKOMMENTIEREN
%\textbf{eCollaboration}\\
%\textbf{Logik und \\Berechenbarkeit}\\
%\textbf{Mathematik für\\ Software Engineering}\\
\end{Huge}
\vspace{10px}
\begin{large}
%% HIER SEMESTER EINGEBEN
Wintersemester 2022/23
\end{large}
\end{flushright}
\end{minipage}
\end{figure}

\vspace{20px}
\begin{huge}
\noindent

%HIER NUMMER DES ÜBUNGSZETTELS EINTRAGEN
\textbf{Übungsblatt 5}
\end{huge}

%HIER DIE JEWEILIGEN AUFGABENNUMMERN UND -NAMEN EINTRAGEN
\pagebreak
\section{Formale Grammatik}
Gegeben ist eine formale Grammatik $g$
\begin{align*}
 G=(V,\Sigma,P,S) V=\{S,A,B,C\} \Sigma = \{e,f,g\}\\
 P = \{ S \longrightarrow AB , A \longrightarrow eB, A \longrightarrow \epsilon , B \longrightarrow BC, B \longrightarrow Bf, B \longrightarrow f \\
 C \longrightarrow gC , C \longrightarrow \epsilon\}\\
 S=S
\end{align*}
\subsection{1.a.i}
Gesucht $\epsilon$
\begin{align*}
 S \underset{S \longrightarrow AB }{\Rightarrow} AB \underset{A \longrightarrow \epsilon }{\Rightarrow} \epsilon B && \text{es kann nicht von }B \longrightarrow \epsilon \text{ abgeleitet werden} && \qquad \vcenter{\hbox{\includegraphics[width=0.08\linewidth]{blitz}}}
\end{align*}
\subsection{1.a.ii}
Gesucht $eff$
\begin{align*}
 S \underset{S \longrightarrow AB}{\Rightarrow} AB \underset{A \longrightarrow eB}{\Rightarrow} eBB \underset{B \longrightarrow f}{\Rightarrow} efB \underset{B \longrightarrow f}{\Rightarrow} eff && \Box
\end{align*}
\subsection{1.a.iii}
Gesucht $eeffg$
\begin{align*}
S \underset{S \longrightarrow AB}{\Rightarrow} AB \underset{A \longrightarrow eB}{\Rightarrow} eBB && \text{es kann nicht von } B \longrightarrow e \text{ abgeleitet werden} && \qquad \vcenter{\hbox{\includegraphics[width=0.08\linewidth]{blitz}}}
\end{align*}
\subsection{1.a.iv}
Gesucht $effffggg$
\begin{align*}
S \underset{S \longrightarrow AB}{\Rightarrow} AB \underset{A \longrightarrow eB}{\Rightarrow} eBB \underset{B \longrightarrow Bf}{\Rightarrow} eBfB \underset{B \longrightarrow Bf }{\Rightarrow} eBffB \underset{B \longrightarrow BC}{\Rightarrow} eBffBC \underset{C \longrightarrow gC }{\Rightarrow} eBffBgC\\
\underset{C \longrightarrow gC}{\Rightarrow} eBffBggC \underset{B \longrightarrow f}{\Rightarrow} efffBggC \underset{B \longrightarrow f}{\Rightarrow} effffggC \underset{C \longrightarrow gC }{\Rightarrow} effffgggC \underset{C \longrightarrow \epsilon}{\Rightarrow} effffggg\epsilon && \Box
\end{align*}
\subsection{1.b}
Diese Grammatik ist keine reguläre Grammatik, da für reguläre Grammatiken $\forall(X-Y) \in P : (X \in V) \wedge (Y \in (\Sigma^* \cup \Sigma^+V))$ gilt. Dies bedeutet, dass $Y$ nur ein Element aus $V$ sein darf und es immer mit $\epsilon$ oder einem anderen Element aus $\Sigma$ beginnen muss. Dies trifft auf
$S \longrightarrow AB ; B \longrightarrow BC ; B \longrightarrow Bf$ nicht zu.
\pagebreak
\section{Deterministische Endliche Automaten}
\begin{figure}[h]
 \begin{center}
  \includegraphics[scale=0.6]{5_2.png}
 \end{center}
\end{figure}
\subsection{2.a Beschreibung als 5-Tupel}
\begin{align*}
 A = (Q=\{q_0,q_1,q_2,q_3\}, \Sigma=\{a,b,c\},\\
 \delta=\{\delta(q_0,a) = \{q_2\},\delta(q_0,b) = \{q_1\},\delta(q_1,a) = \{q_1\},\\
 \delta(q_1,c) = \{q_2\},\delta(q_2,a) = \{q_3\},\delta(q_3,c) = \{q_3\}\}),\\
 q_0 = q_0,F = \{q_3\})
\end{align*}
\subsection{2.b.i}
Wird $baabcacc$ akzeptiert?
\begin{align*}
q_0 \underset{b}{\longrightarrow} q_1 \underset{a}{\longrightarrow} q_1 \underset{a}{\longrightarrow} q_1 \underset{b}{\longrightarrow} \text{toter Zustand} && \qquad \vcenter{\hbox{\includegraphics[width=0.08\linewidth]{blitz}}}
\end{align*}
\subsection{2.b.ii}
Wird $aacc$ akzeptiert?
\begin{align*}
 q_0 \underset{a}{\longrightarrow} q_2 \underset{a}{\longrightarrow} q_3 \underset{c}{\longrightarrow} q_3 \underset{c}{\longrightarrow} q_3 && q_3 \in F && \Box
\end{align*}
\subsection{2.b.iii}
Wird $baacacc$ akzeptiert?
\begin{align*}
 q_0 \underset{b}{\longrightarrow} q_1 \underset{a}{\longrightarrow} q_1 \underset{a}{\longrightarrow} q_1 \underset{c}{\longrightarrow} q_2 \underset{a}{\longrightarrow} q_3 \underset{c}{\longrightarrow} q_3 \underset{c}{\longrightarrow} q_3 && q_3 \in F && \Box
\end{align*}
\subsection{2.b.iv}
Wird $baaac$ akzeptiert?
\begin{align*}
 q_0 \underset{b}{\longrightarrow} q_1 \underset{a}{\longrightarrow} q_1 \underset{a}{\longrightarrow} q_1 \underset{a}{\longrightarrow} q_1 \underset{c}{\longrightarrow} q_2 && q_2 \not \in F && \qquad \vcenter{\hbox{\includegraphics[width=0.08\linewidth]{blitz}}}
\end{align*}
\pagebreak
\subsection{2.c}
$\Sigma=\{a,c\}$ $L_B := \{w \in \Sigma^* | w $ beginnt mit beliebig vielen $a$'s gefolgt von einer ungeraden Anzahl $c$'s (zwischen oder nach den $c$'s können auch wieder $a$'s folgen) $\}$
\begin{figure}[ht] % ’ht’ tells LaTeX to place the figure ’here’ or at the top of the page
\centering % centers the figure
\begin{tikzpicture} [node distance = 3cm, on grid, auto]

\node (A) [state, initial , initial text = {}] {$A$};
\node (B) [state, right = of A] {$B$};
\node (C) [state, accepting, right = of B] {$C$};

\path [-stealth, thick]
    (A) edge node {c} (B)
    (B) edge node {c} (A)
    (A) edge [loop above]  node {a}()
    (B) edge node {a}(C);
\end{tikzpicture}
\end{figure}\\
Überprüfung, $aca$,$aaccca$ und $accaaaca$ sollen akzeptiert werden.
\begin{align*}
A \underset{a}{\longrightarrow} A \underset{c}{\longrightarrow} B \underset{a}{\longrightarrow} C && C \in F && \Box\\
A \underset{a}{\longrightarrow} A \underset{a}{\longrightarrow} A \underset{c}{\longrightarrow} B \underset{c}{\longrightarrow} A \underset{c}{\longrightarrow} B \underset{a}{\longrightarrow} C && C \in F && \Box \\
A \underset{a}{\longrightarrow} A \underset{c}{\longrightarrow} B \underset{c}{\longrightarrow} A \underset{a}{\longrightarrow} A \underset{a}{\longrightarrow} A \underset{a}{\longrightarrow} A \underset{c}{\longrightarrow} B \underset{a}{\longrightarrow} C && C \in F && \Box
\end{align*}
\begin{align*}
 B=(Q=\{A,B,C\}, \Sigma= \{a,c\},\\
 \delta = \{\delta(A,a) = \{A\} , \delta(A,c) = \{B\}\\
 \delta(B,a) = \{C\},\delta(B,c) = \{A\}\}\\
 q_0=A,F=\{C\})
\end{align*}
\subsection{2.d}
\begin{align*}
 B=(V=\{A,B,C\},\Sigma=\{a,c\}\\
 P=\{A \longrightarrow aA, A \longrightarrow cB\\
 B \longrightarrow aC,B \longrightarrow cA\\
 C \longrightarrow \epsilon\},S=A)
\end{align*}
\pagebreak
\section{Nicht-Deterministische Endliche Automaten}
\subsection{3.a Mit der Potennzmengenkonstruktion}
Ausgangs-regeln:
\begin{table}[h]
\begin{center}
 \begin{tabular}{l|c|c}
& $a$ & $b$\\
\hline
$q_0$ & $\{q_0,q_1\}$ & $q_1$ \\
\hline
$q_1$ & $q_2$ & $\{q_0,q_1\}$ \\
\hline
$q_2$ & $q_2$ & $q_0$ \\
 \end{tabular}
\end{center}
\end{table}\\
Daraus können wir folgende Regeln für $A$ erstellen:
\begin{table}[h]
\begin{center}
 \begin{tabular}{l|c|c}
& $a$ & $b$\\
\hline
$q_0$ & $\{q_0,q_1\}$ & $q_1$ \\
\hline
$\{q_0,q_1\}$ & $\{q_0,q_1,q_2\}$ & $\{q_0,q_1\}$ \\
\hline
$\{q_0,q_1,q_2\}$ & $\{q_0,q_1,q_2\}$ & $\{q_0,q_1\}$ \\
\hline
$q_1$ & $q_2$ & $\{q_0,q_1\}$ \\
\hline
$q_2$ & $q_2$ & $q_0$
 \end{tabular}
\end{center}
\end{table}\\
Daraus kann man folgende Tupel erstellen:
\begin{align*}
 A=(Q=\{X,Y,Z,B,C\},\Sigma=\{a,b\},\\ \delta = \{ \delta(X,a) = \{C\},\delta(X,b) = \{Y\},\delta(Y,a) = \{Z\},\delta(Y,b) = \{C\},\delta(Z,a) = \{Z\},\delta(Z,b) = \{X\},\\ \delta(C,a) = \{B\},\delta(C,b) = \{C\},\delta(B,a) = \{b\},\delta(B,b) = \{C\},\\
 q_0 = X , F = \{Z,B\} \})
\end{align*}
\pagebreak
\subsection{3.a Graphisch}
Für eine einfachere Darstellung werden in folgender Grafik die oben agebenen Punkte folgendermaßen umbenannt:
\begin{align*}
 q_0 \longrightarrow X\\
 q_1 \longrightarrow Y\\
 q_2 \longrightarrow Z \\
 \{q_0,q_1\} \longrightarrow C \\
 \{q_0,q_1,q_2\} \longrightarrow B
\end{align*}
\begin{figure}[ht] % ’ht’ tells LaTeX to place the figure ’here’ or at the top of the page
\centering % centers the figure
\begin{tikzpicture} [node distance = 3cm, on grid, auto]

\node (X) [state, initial , initial text = {}] {$X$};
\node (Y) [state, below = of X] {$Y$};
\node (Z) [state,accepting, right = of X] {$Z$};
\node (C) [state, below = of Z] {$C$};
\node (B) [state, accepting, right = of C] {$B$};


\path [-stealth, thick]
    (X) edge node {a} (C)
    (X) edge node {b} (Y)
    (Y) edge node {a} (Z)
    (Y) edge node {b} (C)
    (Z) edge [loop above]  node {a}
    (Z) edge node {b} (X)
    (C) edge node {a} (B)
    (C) edge [loop below]  node {b}()
    (B) edge node {b} (C)
    (B) edge [loop below]  node {a}();
\end{tikzpicture}
\end{figure}
\subsection{3.b.i}
Wird $bbabaa$ akzeptiert ?
\begin{align*}
 X \underset{b}{\longrightarrow} Y \underset{b}{\longrightarrow} C \underset{a}{\longrightarrow} B \underset{b}{\longrightarrow} C \underset{a}{\longrightarrow} B \underset{a}{\longrightarrow} B && B \in F && \Box && \text{Anhand von } A\\
 q_0 \underset{b}{\longrightarrow} q_1 \underset{b}{\longrightarrow} q_0 \underset{a}{\longrightarrow} q_1 \underset{b}{\longrightarrow} q_1 \underset{a}{\longrightarrow} q_2 \underset{a}{\longrightarrow} q_2 && q_2 \in F && \Box && \text{Anhand von } N
\end{align*}
\subsection{3.b.ii}
Wird $bbabb$ akzeptiert ?
\begin{align*}
 X \underset{b}{\longrightarrow} Y \underset{b}{\longrightarrow} C \underset{a}{\longrightarrow} B \underset{b}{\longrightarrow} C \underset{b}{\longrightarrow} C  && C \not \in F && \qquad \vcenter{\hbox{\includegraphics[width=0.08\linewidth]{blitz}}} && \text{Anhand von } A\\
 q_0 \underset{b}{\longrightarrow} q_1 \underset{b}{\longrightarrow} q_1 \underset{a}{\longrightarrow} q_2 \underset{b}{\longrightarrow} q_0 \underset{b}{\longrightarrow} q_1  && q_1 \not \in F && \qquad \vcenter{\hbox{\includegraphics[width=0.08\linewidth]{blitz}}} && \text{Anhand von } N\\
 \text{Ich konnte kein Beispiel in }N\text{ finden, welches akzeptiert wurde.} && &&
\end{align*}
\subsection{3.b.iii}
Wird $abaab$ akzeptiert ?
\begin{align*}
 X \underset{a}{\longrightarrow} C \underset{b}{\longrightarrow} C \underset{a}{\longrightarrow} B \underset{a}{\longrightarrow} B \underset{b}{\longrightarrow} C   && C \not \in F && \qquad \vcenter{\hbox{\includegraphics[width=0.08\linewidth]{blitz}}} && \text{Anhand von } A\\
 q_0 \underset{a}{\longrightarrow} q_0 \underset{b}{\longrightarrow} q_1 \underset{a}{\longrightarrow} q_2 \underset{a}{\longrightarrow} q_2 \underset{b}{\longrightarrow} q_0   && q_0 \not \in F && \qquad \vcenter{\hbox{\includegraphics[width=0.08\linewidth]{blitz}}} && \text{Anhand von } N\\
 \text{Ich konnte kein Beispiel in }N\text{ finden, welches akzeptiert wurde.} && &&
\end{align*}
\subsection{3.b.iv}
Wird $aaaa$ akzeptiert ?
\begin{align*}
 X \underset{a}{\longrightarrow} C \underset{a}{\longrightarrow} B \underset{a}{\longrightarrow} B \underset{a}{\longrightarrow} B  && B \in F && \Box && \text{Anhand von } A \\
  q_0 \underset{a}{\longrightarrow} q_0 \underset{a}{\longrightarrow} q_0 \underset{a}{\longrightarrow} q_1 \underset{a}{\longrightarrow} q_2  && q_2 \in F && \Box && \text{Anhand von } N
\end{align*}
\section{$\epsilon$-Nicht-Deterministische Endliche Automaten}
\subsection{4.a}
Wird $bab$ akzeptiert?
\begin{align*}
 q_0 \underset{\epsilon}{\Rightarrow} q_1 \underset{b}{\Rightarrow} q_0 \underset{\epsilon}{\Rightarrow} q_1 \underset{a}{\Rightarrow} q_1 \underset{b}{\Rightarrow} q_3 && q_3 \in F && \Box
\end{align*}
\subsection{4.b}
Wird $aa$ akzeptiert?
\begin{align*}
 q_0 \underset{\epsilon}{\Rightarrow} q_1 \underset{a}{\Rightarrow} q_2 \underset{a}{\Rightarrow} q_3 && q_3 \in F && \Box
\end{align*}
\subsection{4.c}
Wird $bbbb$ akzeptiert?
\begin{align*}
 q_0 \underset{\epsilon}{\Rightarrow} q_1 \underset{b}{\Rightarrow} q_0 \underset{\epsilon}{\Rightarrow} q_1 \underset{b}{\Rightarrow} q_0 \underset{\epsilon}{\Rightarrow} q_1 \underset{b}{\Rightarrow} q_0 \underset{\epsilon}{\Rightarrow} q_1 \underset{b}{\Rightarrow} q_3 && q_3 \in F && \Box
\end{align*}
\subsection{4.d}
Wird $bbbb$ akzeptiert?
\begin{align*}
 q_0 \underset{\epsilon}{\Rightarrow} q_1 \underset{b}{\Rightarrow} q_0 \underset{\epsilon}{\Rightarrow} q_1 \underset{b}{\Rightarrow} q_0 \underset{\epsilon}{\Rightarrow} q_1 \underset{b}{\Rightarrow} q_3 && q_3 \in F && \Box
\end{align*}
\end{document}
