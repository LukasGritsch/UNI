% !TEX encoding = UTF-8 Unicode
\documentclass[10pt,ngerman]{scrartcl}
\usepackage{url,bm,tikz,a4wide}
\usepackage[utf8]{inputenc}
\usepackage{booktabs}
\usepackage{amsmath,amssymb}
\usepackage[ngerman]{babel}
\usepackage{graphicx,tikzsymbols}
\usepackage{tikz}
\usepackage{float}
\usepackage{graphicx}
\usepackage{arev}

\renewcommand{\theenumi}{\arabic{paragraph}.\alph{enumi}}
\renewcommand{\theenumii}{\roman{enumii}}
\renewcommand{\labelenumi}{\roman{enumi})}

\def\firstcircle{(90:1.75cm) circle (2.5cm)}
\def\secondcircle{(210:1.75cm) circle (2.5cm)}
\def\thirdcircle{(330:1.75cm) circle (2.5cm)}

%SPEZIELLE KOMMENTARE FÜR LOGIK UND BERECHENBARKEIT
\newcommand{\w }{\texttt{ W }}
\newcommand{\f }{\texttt{ F }}

\setcounter{secnumdepth}{-1}


\begin{document}

\begin{figure}[htbp]
\begin{minipage}[b]{0.50\linewidth}
\begin{Large}

%HIER PERSÖNLICHE DATEN EINTRAGEN
	\textbf{Name:}\\
	Gritsch 			\\
	\textbf{Vorname:}\\
	Lukas 				\\
	\textbf{Matrikelnummer:}\\
	2210836017

\end{Large}
\end{minipage}
\begin{minipage}[b]{0.50\linewidth}
\begin{flushright}
\begin{Huge}
%% HIER LEHRVERANSTALTUNG EINGEBEN/EINKOMMENTIEREN
%\textbf{eCollaboration}\\
%\textbf{Logik und \\Berechenbarkeit}\\
%\textbf{Mathematik für\\ Software Engineering}\\
\end{Huge}
\vspace{10px}
\begin{large}
%% HIER SEMESTER EINGEBEN
Wintersemester 2022/23
\end{large}
\end{flushright}
\end{minipage}
\end{figure}

\vspace{20px}
\begin{huge}
\noindent

%HIER NUMMER DES ÜBUNGSZETTELS EINTRAGEN
\textbf{Übungsblatt 4}
\end{huge}

%HIER DIE JEWEILIGEN AUFGABENNUMMERN UND -NAMEN EINTRAGEN
\pagebreak
\section{Relationen 1}
\begin{align*}
D = \{\epsilon,\omega, \theta, \zeta\}, I = \{\alpha, \beta,\gamma, \delta\}, B = \{\spadesuit, \varheart, \clubsuit, \diamondsuit\}, S = \{1, 2, 3, 4\}, E = \{5, 4, 3, 2, 1\},\\
M = \{a, b, c, d\}, C = \{b, c, d, e\}, T = \{b, d, f , g\}
\end{align*}
\subsection{1.a)}
$U \subseteq E\text{ x }E := \{(1, 5), (2, 4), (3, 3), (4, 2), (5, 5)\}$
\begin{align*}
E\textbf{ x }E := \{\color{blue}(5,5)\color{black},(5,4),(5,3),(5,2),(5,1),\\
(4,5),(4,4),(4,3),\color{blue}(4,2)\color{black},(4,1),\\
(3,5),(3,4),\color{blue}(3,3)\color{black},(3,2),(3,1),\\
(2,5),\color{blue}(2,4)\color{black},(2,3),(2,2),(2,1),\\
\color{blue}(1,5)\color{black},(1,4),(1,3),(1,2),(1,1)\}
\end{align*}
$U$ ist eine Teilmenge von $E\text{ x }E$
\begin{align*}
\text{reflexiv}: \Leftrightarrow \underset{x \in E}{\forall} (x,x) \in U && (1,1) \text{ fehlt}&& \qquad \vcenter{\hbox{\includegraphics[width=0.08\linewidth]{blitz}}}\\
\text{symetrisch}: \Leftrightarrow \underset{x,y \in E}{\forall} (x,y) \in U \Rightarrow (y,x) \in U && (5,1) \text{ fehlt}&& \qquad \vcenter{\hbox{\includegraphics[width=0.08\linewidth]{blitz}}}\\
\text{transitiv}: \Leftrightarrow \underset{x,y,z \in E}{\forall} ((x,y) \in U \wedge (y,x) \in U) \Rightarrow (x,z) \in U && (5,1) \wedge (1,4)\text{ fehlen}&& \qquad \vcenter{\hbox{\includegraphics[width=0.08\linewidth]{blitz}}}
\end{align*}
\subsection{1.b)}
$Q \subseteq M\text{ x }C := \{(a,c),(b,e),(d,d),(a,e),(c,a)\}$
\begin{align*}
 M\text{ x }C := \{(a,b),\color{blue}(a,c)\color{black},(a,d),\color{blue}(a,e)\color{black},\\
 (b,b),(b,c),(b,d),\color{blue}(b,e)\color{black},\\
 (c,b),(c,c),(c,d),(c,e),\\
 (d,b),(d,c),\color{blue}(d,d)\color{black},(d,e)\}
\end{align*}
$Q$ ist keine Teilmenge von $M\text{ x }C$ also $Q \not\subseteq M\text{ x }C$ da $ (c,a) \notin M\text{ x }C$ ist.
\pagebreak
\subsection{1.c)}
$R \subseteq S\text{ x }E := \{(1, 1), (1, 4), (1, 3), (2, 3), (2, 4), (3, 3), (4, 4), (4, 3)\}$
\begin{align*}
S\text{ x }E := \{(1,5),\color{blue}(1,4)\color{black},\color{blue}(1,3)\color{black},(1,2),\color{blue}(1,1)\color{black},\\
(2,5),\color{blue}(2,4)\color{black},\color{blue}(2,3)\color{black},(2,2),(2,1),\\
(3,5),(3,4),\color{blue}(3,3)\color{black},(3,2),(3,1),\\
(4,5),\color{blue}(4,4)\color{black},\color{blue}(4,3)\color{black},(4,2),(4,1)\}
\end{align*}
$R$ ist eine Teilmenge von $S\text{ x }E $\\
Wir haben in der Vorlesung refelxiv, symetrisch und transitiv Definiert als Relation \textbf{auf eine Menge}. Da dies hier nicht der Fall ist kann ich dies für diese Relation nicht überprüfen.
\subsection{1.d)}
$Z \subseteq D\text{ x }D := \{(\epsilon, \epsilon), (\omega, \omega), (\theta, \theta), (\zeta, \zeta)\}$
\begin{align*}
 D\text{ x }D :=\color{blue} \{(\epsilon,\epsilon)\color{black},(\epsilon,\omega),(\epsilon,\theta),(\epsilon,\zeta),\\
 (\omega,\epsilon),\color{blue}(\omega,\omega)\color{black},(\omega,\theta),(\omega,\zeta),\\
 (\theta,\epsilon),(\theta,\omega),\color{blue}(\theta,\theta)\color{black},(\theta,\zeta),\\
 (\zeta,\epsilon),(\zeta,\omega),(\zeta,\theta),\color{blue}(\zeta,\zeta)\color{black}\}
\end{align*}
$Z$ ist Teilmenge von $D\text{ x }D $
\begin{align*}
\text{reflexiv}: \Leftrightarrow \underset{x \in D}{\forall} (x,x) \in Z && \text{ Wahr}&&\Box\\
\text{symetrisch}: \Leftrightarrow \underset{x,y \in D}{\forall} (x,y) \in Z \Rightarrow (y,x) \in Z && \text{ Wahr}&& \Box\\
\text{transitiv}: \Leftrightarrow \underset{x,y,z \in D}{\forall} ((x,y) \in Z \wedge (y,x) \in Z) \Rightarrow (x,z) \in Z && \text{Wahr} && \Box
\end{align*}
\section{2}
\subsection{2.a)}
Ausgangmenge = $\{(1, 2), (3, 4), (d, d), (a, a), (2, 1), (c, 2), (d, 2), (2, c)\}$\\
Gefordert: Antisymmetrie
\begin{align*}
 antisymmetrie:\Leftrightarrow \underset{x,y \in M}{\forall} ((x,y) \in M^2 \wedge (y,x) \in M^2) \Rightarrow x = y\\
 \{(1, 2), (3, 4), (d, d), (a, a), (c, 2), (d, 2)\}
\end{align*}
\subsection{2.b)}
Ausgangmenge = $\{(a, 1), (b, 1), (1, 1), (1, b), (1, a), (2, 2), (b, c), (c, b), (a, 4)\}$\\
Gefordert: Irreflexivität
\begin{align*}
 \text{irreflexivität} :\Leftrightarrow \underset{x \in M}{\forall} (x,x) \notin M^2\\
 {(a, 1), (b, 1), (1, b), (1, a), (b, c), (c, b), (a, 4)}
\end{align*}
\subsection{2.c)}
Ausgangmenge : $\{(1, d), (2, c), (3, b), (4, a), (a, a), (c, 2), (1, 3), (2, 4)\}$\\
Gefordert: Asymmetrie
\begin{align*}
asymmetrie: \Leftrightarrow \underset{x,y \in M}{\forall} (x,y) \in M^2 \Rightarrow (y,x) \notin M^2\\
 \{(1, d), (2, c), (3, b), (4, a), (1, 3), (2, 4)\}
\end{align*}
\section{3}
$M = \{\spadesuit, x, Kiwi, \{\Omega, 5\}, \zeta, z\}$
\subsection{3.a)}
\begin{align*}
 \underset{A \in P_a}{\forall} |A| \leq 3\\
 P_a := \{\{x,\spadesuit,Kiwi\},\{\{\Omega, 5\}\},\{\zeta,z\}\}
\end{align*}
\subsection{3.b)}
\begin{align*}
 |P_b| = 4\\
 P_b := \{\{x\},\{Kiwi\},\{\{\Omega, 5\}\},\{\spadesuit,\zeta,z\}\}
\end{align*}
\subsection{3.c)}
\begin{align*}
 \underset{A \in P_c}{\exists!}|A| = 3\\
 P_c := \{\{x\},\{Kiwi\},\{\spadesuit\},\{\{\Omega, 5\},\zeta, z\}\}
\end{align*}
\section{4}
$M = \{a, b, c, d, e, f , g\}$
\subsection{4.a)}
\begin{align*}
 R \subseteq M \textbf{ x } M :=
\{(f , f ), (b, g), (a, b), (a, a), (b, d), (e, c), (e, e),\\
(d, g), (d, d), (b, a), (g, b), (d, a), (c, e), (c, c),\\
(d, b), (a, g), (b, b), (g, g), (g, a), (a, d), (g, d)\}
\end{align*}
Äquivalenzrelation heißt, dass eine Relation transitiv,refelxiv und symetrisch ist.
\begin{align*}
\text{reflexiv}: \Leftrightarrow \underset{x \in M}{\forall} (x,x) \in R && \text{ Wahr}&&\Box\\
\end{align*}
da die Tupel $(a,a),(b,b),(c,c),(d,d),(e,e),(f,f),(g,g)$ enthalten sind
\begin{align*}
\text{symetrisch}: \Leftrightarrow \underset{x,y \in M}{\forall} (x,y) \in R \Rightarrow (y,x) \in R && \text{ Wahr}&& \Box
\end{align*}
da die Tupel $(b,g),(g,b)$ und $(a,b),(b,a)$ und $(b,d),(d,b)$ und $(e,c),(c,e)$ und $(d,g),(g,d)$ und $(d,a),(a,d)$ und $(a,g),(g,a)$ enthalten sind.
\begin{align*}
\text{transitiv}: \Leftrightarrow \underset{x,y,z \in M}{\forall} ((x,y) \in R \wedge (y,x) \in R) \Rightarrow (x,z) \in R && \text{ Wahr}&& \Box
\end{align*}
da folgende Aussagen wahr sind:
\begin{align*}
 ((b,g) \in R \wedge (g,b) \in R) \Rightarrow (b,d)\\
 ((a,b) \in R \wedge (b,a) \in R) \Rightarrow (a,d)\\
 ((b,d) \in R \wedge (d,b) \in R) \Rightarrow (b,g)\\
 ((e,c) \in R \wedge (c,e) \in R) \Rightarrow (e,e)\\
 ((d,g) \in R \wedge (g,d) \in R) \Rightarrow (d,b)\\
 ((d,a) \in R \wedge (a,d) \in R) \Rightarrow (d,g)\\
 ((a,g) \in R \wedge (g,a) \in R) \Rightarrow (a,b)
\end{align*}
\subsection{4.b)}
\begin{align*}
[a]_R = \{b,a,d,g\}\\
[b]_R = \{g,a,d,b\}\\
[c]_R = \{e,c\}\\
[d]_R = \{b,g,d,a\}\\
[e]_R = \{c,e\}\\
[f]_R = \{f\}\\
[g]_R = \{b,d,a,g\}
\end{align*}
Da $[a]_R = [b]_R = [d]_R = [g]_R$ und $[c]_R = [e]_R$ komme ich auf folgende Lösung:
\begin{align*}
 [a]_R = \{b,a,d,g\} = [b]_R = [d]_R = [g]_R\\
 [c]_R = \{e,c\} = [e]_R\\
 [f]_R = \{f\}
\end{align*}
\section{5}
\begin{align*}
 M = \{\alpha, \beta, \gamma, \delta\}, N = \{\spadesuit, \varheart, \clubsuit, \diamondsuit\}, O = \{\Omega, \Delta, \Sigma, \Pi\}, P = \{1, 3, 5, 7\}, \\
 Q = \{2, 4, 6, 8\}, R = \{9, 10, 11\}, S = \{a, b, c, d\}, T = \{e, f , g, h\}, U = \{i, j, k \}
\end{align*}
\subsection{5.a)}
$f_1: M\longrightarrow O := \{(\alpha, \Omega), (\alpha, \Omega), (\beta, \Pi), (\delta, \Sigma), (\gamma,\varheart)\} $\\
Dies ist weder eine partielle noch eine totale Funktion, da das Tupel $(\gamma,\varheart)$ nicht in $M\text{ x }O$ enthalten ist.
\subsection{5.b)}
$f_2: N \longrightarrow M := \{(\varheart, \alpha), (\spadesuit, \gamma), (\diamondsuit, \delta), (\clubsuit, \gamma)\}$\\
Dies ist eine totale Funktion, da für jedes Element aus N ein Element in M existiert also $\underset{x \in N}{\forall} \underset{y \in M}{\exists} (x,y) \in f$
\begin{align*}
injektiv :\Leftrightarrow \underset{x_1,x_2 \in N}{\forall} f (x_1) = f (x_2) | \Rightarrow x_1 = x_2 \\ f(\spadesuit) = f(\clubsuit) | \Rightarrow \spadesuit = \clubsuit && \qquad \vcenter{\hbox{\includegraphics[width=0.08\linewidth]{blitz}}}
\end{align*}
Nicht injektiv.
\begin{align*}
surjektiv :\Leftrightarrow \underset{y \in M}{\forall} \underset{x \in N}{\exists} f(x) = y && \qquad \vcenter{\hbox{\includegraphics[width=0.08\linewidth]{blitz}}} \\
\text{Es gibt kein x welches für welches }f(x) = \beta \text{ gilt.}
\end{align*}
Nicht surjektiv.
\begin{align*}
 bijektiv :\Leftrightarrow injektiv (f ) \wedge surjektiv (f )
\end{align*}
Nicht bijektiv.
\subsection{5.c)}
$f_3: T \longrightarrow U := \{(f , j), (g, k ), (h, j), (e, i)\}$\\
Dies ist eine totale Funktion, da für jedes Element aus T ein Element in U existiert also $\underset{x \in T}{\forall} \underset{y \in U}{\exists} (x,y) \in f$
\begin{align*}
injektiv :\Leftrightarrow \underset{x_1,x_2 \in N}{\forall} f (x_1) = f (x_2) | \Rightarrow x_1 = x_2 \\ f(f) = f(h) | \Rightarrow f = h && \qquad \vcenter{\hbox{\includegraphics[width=0.08\linewidth]{blitz}}}
\end{align*}
Nicht injektiv.
\begin{align*}
surjektiv :\Leftrightarrow \underset{y \in M}{\forall} \underset{x \in N}{\exists} f(x) = y && \Box
\end{align*}
Diese Funktion ist surjektiv.
\begin{align*}
 bijektiv :\Leftrightarrow injektiv (f ) \wedge surjektiv (f )
\end{align*}
Nicht bijektiv, da die Injektivität nicht erfüllt ist.
\subsection{5.d}
$f_4: R \longrightarrow Q := \{(9, 9), (10, 8), (11, 6)\}$\\
Dies ist weder eine partielle noch eine totale Funktion, da das Tupel $(9,9)$ nicht in $R \text{ x } Q$ enthalten ist.
\pagebreak
\section{6}
\subsection{6.a)}
$f(x) := \begin{cases}
3 * f(x - 1) +2 & \text{falls } x \geq 1 \\
0 & \, \text{sonst}
\end{cases}$
\setcounter{equation}{0}
\begin{align}
f(1) = 3 * f(0) + 2\\
f(1) = 3 * 0 + 2\\
f(1) = 2
\end{align}
\begin{align}
f(2) = 3 * f(1) + 2\\
f(2) = 3 * 2 + 2\\
f(2) = 8
\end{align}
\begin{align}
f(3) = 3 * f(3) + 2\\
f(3) = 3 * 8 + 2\\
f(3) = 26
\end{align}
\begin{align}
f(4) = 3 * f(3) + 2\\
f(4) = 3 * 26 + 2\\
f(4) = 80
\end{align}
\begin{align}
f(5) = 3 * f(4) + 2\\
f(5) = 3 * 80 + 2\\
f(5) = 242
\end{align}
\subsection{6.b)}
$f(x) := \begin{cases}
f(x - 2) + 2x & \text{falls } x > 1\\
1 & \text{sonst}
\end{cases}$
\setcounter{equation}{0}
\begin{align}
f(1) = 1
\end{align}
\begin{align}
f(2) = f(0) + 2 * 2\\
f(2) = 1+4\\
f(2) = 5
\end{align}
\begin{align}
f(3) = f(1) + 2 * 3\\
f(3) = 1+6\\
f(3) = 7
\end{align}
\begin{align}
f(4) = f(2) + 2 * 4\\
f(4) = 5+8\\
f(4) = 13
\end{align}
\begin{align}
f(5) = f(3) + 2 * 5\\
f(5) = 7+10\\
f(5) = 17
\end{align}
\end{document}
