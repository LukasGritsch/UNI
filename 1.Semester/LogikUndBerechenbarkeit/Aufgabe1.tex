% !TEX encoding = UTF-8 Unicode
\documentclass[10pt,ngerman]{scrartcl}
\usepackage{url,bm,tikz,a4wide}
\usepackage[utf8]{inputenc}
\usepackage{booktabs}
\usepackage{amsmath,amssymb}
\usepackage[ngerman]{babel}
\usepackage{graphicx,tikzsymbols}
\usepackage{tikz}
\usepackage{float}

\renewcommand{\theenumi}{\arabic{paragraph}.\alph{enumi}}
\renewcommand{\theenumii}{\roman{enumii}}
\renewcommand{\labelenumi}{\roman{enumi})}

%SPEZIELLE KOMMENTARE FÜR LOGIK UND BERECHENBARKEIT
\newcommand{\w }{\texttt{W}}
\newcommand{\f }{\texttt{F}}

\setcounter{secnumdepth}{-1}


\begin{document}

\begin{figure}[htbp]
\begin{minipage}[b]{0.50\linewidth}
\begin{Large}

%HIER PERSÖNLICHE DATEN EINTRAGEN
	\textbf{Name:}\\
	Gritsch 			\\
	\textbf{Vorname:}\\
	Lukas 				\\
	\textbf{Matrikelnummer:}\\
	2210836017

\end{Large}
\end{minipage}
\begin{minipage}[b]{0.50\linewidth}
\begin{flushright}
\begin{Huge}
%% HIER LEHRVERANSTALTUNG EINGEBEN/EINKOMMENTIEREN
%\textbf{eCollaboration}\\
%\textbf{Logik und \\Berechenbarkeit}\\
%\textbf{Mathematik für\\ Software Engineering}\\
\end{Huge}
\vspace{10px}
\begin{large}
%% HIER SEMESTER EINGEBEN
Wintersemester 2022/23
\end{large}
\end{flushright}
\end{minipage}
\end{figure}

\vspace{20px}
\begin{huge}
\noindent

%HIER NUMMER DES ÜBUNGSZETTELS EINTRAGEN
\textbf{Übungsblatt 1}
\end{huge}

%HIER DIE JEWEILIGEN AUFGABENNUMMERN UND -NAMEN EINTRAGEN
\pagebreak
\section{1.a)}

Um die Aussage $A = (y\wedge(\neg x)) \Rightarrow z$ in einer Wahrheitstabelle darstellen zu könne brauch man ein Tabelle mit mindestens 4 Spalten und 8 Zeilen
\begin{table}[h]
\begin{center}

\begin{tabular}{l|c|c|c|c|c|c}
ZeilenNr & x &y &z & $\neg x$ &y $\wedge (\neg x)$ & A \\
\hline
&&&&&&\\
%   x   y   z   !x  f   A
1 & F & F & F & W & F & W\\
2 & F & F & W & W & F & W\\
3 & F & W & F & W & W & F\\
4 & W & F & F & F & F & W\\
5 & W & W & F & F & F & W\\
6 & W & F & W & F & F & W\\
7 & F & W & W & W & W & W\\
8 & W & W & W & F & F & W\\
\end{tabular}
\end{center}
\end{table}
\subsection{1.a) Beispiel als Syntaxbaum}
\begin{center}
\begin{tikzpicture}
\filldraw[black] (0,0) circle (0pt) node[anchor=center]{$\Rightarrow$};
\draw[gray, thick] (0,-0.2) -- (1,-1);
\draw[gray, thick] (0,-0.2) -- (-1,-1);
\filldraw[black] (1,-1.2) circle (0pt) node[anchor=center]{$z$};
\filldraw[black] (-1,-1.2) circle (0pt) node[anchor=center]{$\wedge$};
\draw[gray, thick] (-1,-1.4) -- (-2,-2.2);
\draw[gray, thick] (-1,-1.4) -- (0,-2.2);
\filldraw[black] (-2,-2.4) circle (0pt) node[anchor=center]{$y$};
\filldraw[black] (0,-2.4) circle (0pt) node[anchor=center]{$\neg$};
\draw[gray, thick] (0,-2.6) -- (1,-3.4);
\filldraw[black] (1,-3.6) circle (0pt) node[anchor=center]{$x$};
\end{tikzpicture}
\end{center}

\pagebreak
\section{1.b)}

Um die Aussage $B = (x\vee (\neg y)) \wedge ((\neg x) \Rightarrow z)$ in einer Wahrheitstabelle darstellen zu könne brauch man ein Tabelle mit mindestens 4 Spalten und 8 Zeilen
\begin{table}[h]
\begin{center}
\begin{tabular}{l|c|c|c|c|c|c|c|c}
ZeilenNr & x &y &z & $\neg x$ & $\neg y$ & $x\vee (\neg y)$ & $(\neg x) \Rightarrow z$ & B \\
\hline
&&&&&&&&\\
%   x   y   z  !x  !y   f  f1   B
1 & F & F & F & W & W & W & F & F\\
2 & F & F & W & W & W & W & W & W\\
3 & F & W & F & W & F & F & F & F\\
4 & W & F & F & F & W & W & W & W\\
5 & W & W & F & F & F & W & W & W\\
6 & W & F & W & F & W & W & W & W\\
7 & F & W & W & W & F & F & W & F\\
8 & W & W & W & F & F & W & W & W\\
\end{tabular}
\end{center}
\end{table}
\subsection{1.a) Beispiel als Syntaxbaum}
\begin{center}
\begin{tikzpicture}
\filldraw[black] (0,0) circle (0pt) node[anchor=center]{$\wedge$};
\draw[gray, thick] (0,-0.2) -- (2,-1);
\draw[gray, thick] (0,-0.2) -- (-2,-1);
\filldraw[black] (2,-1.2) circle (0pt) node[anchor=center]{$\Rightarrow$};
\filldraw[black] (-2,-1.2) circle (0pt) node[anchor=center]{$\vee$};
\draw[gray, thick] (-2,-1.4) -- (-3,-2.4);
\draw[gray, thick] (-2,-1.4) -- (-1,-2.4);
\filldraw[black] (-3,-2.6) circle (0pt) node[anchor=center]{$x$};
\filldraw[black] (-1,-2.6) circle (0pt) node[anchor=center]{$\neg$};
\draw[gray, thick] (2,-1.4) -- (3,-2.4);
\draw[gray, thick] (2,-1.4) -- (1,-2.4);
\filldraw[black] (3,-2.6) circle (0pt) node[anchor=center]{$z$};
\filldraw[black] (1,-2.6) circle (0pt) node[anchor=center]{$\neg$};
\draw[gray, thick] (1,-2.8) -- (2,-3.8);
\draw[gray, thick] (-1,-2.8) -- (-2,-3.8);
\filldraw[black] (2,-4) circle (0pt) node[anchor=center]{$x$};
\filldraw[black] (-2,-4) circle (0pt) node[anchor=center]{$y$};
\end{tikzpicture}
\end{center}
\pagebreak
\section{2.a)}
\begin{align}
A = x \vee (x \wedge y ) \Rightarrow \neg x && \text{Distributivität}\\
A = (x \vee x \wedge x \vee y) \Rightarrow \neg x && \text{Idempotenz} \\
A = (x \wedge x \vee y) \Rightarrow \neg x && \text{Idempotenz} \\
A = (x \vee y) \Rightarrow \neg x
\end{align}
\begin{table}[H]
\begin{center}
\begin{tabular}{l|c|c|c|c|c}
ZeilenNr & x &y & $x \vee y$ & $\neg x$ &A \\
\hline
&&&&&\\
%   x   y   f  !x   A
1 & F & F & F & W & W\\
2 & F & W & W & W & W\\
3 & W & F & W & F & F\\
4 & W & W & W & F & F\\
\end{tabular}
\end{center}
\end{table}
Dieser Ausdruck ist erfüllbar.
\section{2.b)}
\setcounter{equation}{0}
\begin{align}
A =(\neg (\neg (x \wedge x))) \vee y && \text{Idempotenz} \\
A =(\neg (\neg x)) && \text{Involution} \\
A = x \vee y
\end{align}
\begin{table}[H]
\begin{center}
\begin{tabular}{l|c|c|c|c|c}
ZeilenNr & x &y & $x \vee y$\\
\hline
&&&\\
%   x   y   f  !x   A
1 & F & F & F\\
2 & F & W & W\\
3 & W & F & W\\
4 & W & W & W\\
\end{tabular}
\end{center}
\end{table}
Dieser Ausdruck ist erfüllbar.
\pagebreak
\section{2.c)}
\setcounter{equation}{0}
\begin{align}
A =x \vee (\neg x \wedge W) && \text{Distributivität}\\
A =x \vee (\neg x) \wedge x \vee W && \text{Invarianz}\\
A =x \vee (\neg x) \wedge W && \text{Komplementarität}\\
A =W \wedge W
\end{align}
Dieser Ausdruck ist eine Tautologie, da diese immer Wahr ist.
\section{2.d)}
\setcounter{equation}{0}
\begin{align}
A =\neg (x \vee (\neg x \wedge W))&& \text{Neutralität}\\
A =\neg (x \vee \neg x)   && \text{Komplementarität}\\
A =\neg(W) \\
A = F
\end{align}
Dieser Ausdruck ist ein Wiederspruch, da diese immer Falsch ist.
\section{3.a)}
\setcounter{equation}{0}
\begin{align}
(x \vee y) \wedge \neg x \equiv (\neg y \Rightarrow x) \wedge (\neg y \vee x)
\end{align}
Für die Darstellung in einer Wahrheitstabelle benötigt man mindestens 3 Spalten und 4 Zeilen
\begin{table}[H]
\begin{center}
\begin{tabular}{l|c|c|c|c|c|c|c|c|c}
ZeilenNr&x&y&$x \vee y$ & $\neg x$ & $(x \vee y) \wedge \neg x $ & $\neg y$ & $\neg y \Rightarrow x$ & $\neg y \vee x$ & $(\neg y \Rightarrow x) \wedge (\neg y \vee x)$\\
\hline
&&&&&&&&&\\
%   x   y   f  !x   a                           !y  f1  f2   b
1 & F & F & F & W & \color{blue}F\color{black} & W & F & W & \color{blue}F\color{black}\\
2 & F & W & W & W & \color{blue}W\color{black} & F & W & F & \color{blue}F\color{black}\\
3 & W & F & W & F & \color{blue}F\color{black} & W & W & W & \color{blue}W\color{black}\\
4 & W & W & W & F & \color{blue}F\color{black} & F & W & W & \color{blue}W\color{black}\\
\end{tabular}
\end{center}
\end{table}
\flushleft Aus der Wahrheitstabelle kann abgelesen werden, dass die Aussage, welche oben getroffen wurde nicht stimmt. Also $ (x \vee y) \wedge \neg x \not \equiv (\neg y \Rightarrow x) \wedge (\neg y \vee x) $
\section{3.b)}
\setcounter{equation}{0}
\begin{align}
\neg y \vee x \equiv \neg (( y \wedge (z \vee \neg z)) \wedge \neg x)&& \text{Komplementatrität}\\
\neg y \vee x \equiv \neg (( y \wedge W ) \wedge \neg x) && \text{Neutralität}\\
\neg y \vee x \equiv \neg ( y \wedge \neg x)  && \text{De Morgan}\\
\neg y \vee x \equiv \neg y \vee x
\end{align}
\flushleft Diese Aussage ist Gleichwertig.
\pagebreak
\section{4.a)}
Der Kettenschluss beasgt, dass wenn $a \Rightarrow b$ Wahr ist und $b \Rightarrow c$ Wahr ist, dann ist auch $a \Rightarrow c$ Wahr.
\begin{table}[H]
\begin{center}
\begin{tabular}{l|c|c|c|c|c|c|}
ZeilenNr & a & b & c & $a \Rightarrow b$ & $b \Rightarrow c$ & $a \Rightarrow c$ \\
\hline
&&&&&&\\
%   a   b   c   f  f1  f2
1 & F & F & F & \color{blue}W\color{black} & \color{blue}W\color{black} & \color{blue}W\color{black}\\
2 & F & F & W & \color{blue}W\color{black} & \color{blue}W\color{black} & \color{blue}W\color{black}\\
3 & F & W & F & W & F & W\\
4 & W & F & F & F & W & F\\
5 & F & W & W & \color{blue}W\color{black} & \color{blue}W\color{black} & \color{blue}W\color{black}\\
6 & W & F & W & F & W & W\\
7 & W & W & F & W & F & F\\
8 & W & W & W & \color{blue}W\color{black} & \color{blue}W\color{black} & \color{blue}W\color{black}\\
\end{tabular}
\end{center}
\end{table}
Aus der Wahrheitstabelle kann man ablesen, dass in jedem Fall in welchem $a \Rightarrow b$ Wahr ist und $b \Rightarrow c$ Wahr ist, $a \Rightarrow c$ auch Wahr ist.
\section{4.b)}
\setcounter{equation}{0}
\begin{align*}
\text{Wenn Peter} \underbrace{\text{Logik mag}}_{\substack{x}}\text{, dann ist} \underbrace{\text{ist er cool.}}_{\substack{y}} && x \Rightarrow y\\
\text{Peter ist}\underbrace{\text{nicht cool.}}_{\substack{\neg y}} && \neg y\\
\text{Also, mag Peter auch} \underbrace{\text{keine Logik.}}_{\substack{\neg x}} && \neg x
\end{align*}
Bei der obigen Aussage handelt es sich um einen Modus Tollens. Denn wenn $ x \Rightarrow y $ Wahr ist und $\neg y$ Wahr ist. Dann ist auch $\neg x$ Wahr.
\section{4.c)}
Folgende Aussage beschreibt einen Modus Polen:
\setcounter{equation}{0}
\begin{align*}
\text{Wenn ich die}\underbrace{\text{Wäsche mache}}_{\substack{x}}\text{, dann ist meine}\underbrace{\text{Freundin zufrieden}}_{\substack{y}}\text{mit mir.} && x \Rightarrow y\\
\text{Ich}\underbrace{\text{mache die Wäsche.}}_{\substack{x}} && x\\
\text{Also ist meine}\underbrace{\text{Freundin auch zufrieden}}_{\substack{y}}\text{mit mir.} && y
\end{align*}
Wenn $x \Rightarrow y$ Wahr ist und $x$ Wahr ist. Dann ist auch $y$ Wahr.
\pagebreak
\section{5)}
\begin{center}
\setcounter{equation}{0}
\begin{align*}
(\exists x f(x,3) +y > g(3,z) *42)
\end{align*}
\begin{tikzpicture}
\filldraw[black] (0,0) circle (0pt) node[anchor=center]{$E$};
\filldraw[black] (3,0) circle (0pt) node[anchor=center]{$..............................Existenzaussage$};
\draw[gray, thick] (0,-0.2) -- (2,-1);
\draw[gray, thick] (0,-0.2) -- (-2,-1);
\filldraw[black] (2,-1.2) circle (0pt) node[anchor=center]{$>$};
\filldraw[black] (-2,-1.2) circle (0pt) node[anchor=center]{$x$};
\filldraw[black] (2,-2.7) circle (0pt) node[anchor=center]{$.....................................$};
\filldraw[black] (4,-1.2) circle (0pt) node[anchor=center]{$...........atomare Aussage$};
\filldraw[black] (-3,-1.2) circle (0pt) node[anchor=center]{$Variable ...$};
\draw[gray, thick] (2,-1.4) -- (4,-2.2);
\draw[gray, thick] (2,-1.4) -- (0,-2.2);
\filldraw[black] (4,-2.6) circle (0pt) node[anchor=center]{$*$};
\filldraw[black] (5.2,-2.6) circle (0pt) node[anchor=center]{$...Funktionen$};
\filldraw[black] (0,-2.6) circle (0pt) node[anchor=center]{$+$};
\filldraw[black] (3,-4.2) circle (0pt) node[anchor=center]{$Konstante$};
\draw[gray, thick] (4,-2.8) -- (3,-3.6);
\draw[gray, thick] (4,-2.8) -- (5,-3.6);
\filldraw[black] (3,-3.8) circle (0pt) node[anchor=center]{$42$};
\filldraw[black] (-2.2,-3.8) circle (0pt) node[anchor=center]{$Term.............$};
\filldraw[black] (5,-3.8) circle (0pt) node[anchor=center]{$g$};
\filldraw[black] (6,-3.8) circle (0pt) node[anchor=center]{$........Term$};
\filldraw[black] (1,-4.2) circle (0pt) node[anchor=center]{$Variable$};
\draw[gray, thick] (0,-2.8) -- (-1,-3.6);
\draw[gray, thick] (0,-2.8) -- (1,-3.6);
\filldraw[black] (1,-3.8) circle (0pt) node[anchor=center]{$y$};
\filldraw[black] (-1,-3.8) circle (0pt) node[anchor=center]{$f$};
\draw[gray, thick] (-1,-4) -- (-2,-4.8);
\draw[gray, thick] (-1,-4) -- (0,-4.8);
\filldraw[black] (-2,-5) circle (0pt) node[anchor=center]{$x$};
\filldraw[black] (0,-5) circle (0pt) node[anchor=center]{$3$};
\draw[gray, thick] (5,-4) -- (4,-4.8);
\draw[gray, thick] (5,-4) -- (6,-4.8);
\filldraw[black] (4,-5) circle (0pt) node[anchor=center]{$3$};
\filldraw[black] (6,-5) circle (0pt) node[anchor=center]{$z$};
\filldraw[black] (7,-5) circle (0pt) node[anchor=center]{$...Variable$};
\filldraw[black] (4,-5.5) circle (0pt) node[anchor=center]{$Konstante$};
\filldraw[black] (0,-5.5) circle (0pt) node[anchor=center]{$Konstante$};
\filldraw[black] (-3,-5) circle (0pt) node[anchor=center]{$Variable...$};
\end{tikzpicture}
\end{center}
\section{6.a)}
Der Zusammenhang \textit{istHauptstadtVon(Paris,Frankreich)} kann als
\begin{itemize}
 \item Prädikat interpretiert werden. Welches den Zustand $W$ einnimmt
 \item 2-Stellige Funktion mit dem Rückgabewert $W$ interpretiert werden.
\end{itemize}
Meiner Ansicht nach sind beide Interpretationen richtig. Wenn ich mich für eine Variante entscheiden müsste, dann würde ich es als \textbf{Prädikat} interpretieren.
\section{6.b)}
Der Zusammenhang $*(5,15,2,4)$ ist eine 4-Stellige Funktion mit dem Rückgabewert $600$
\section{6.c)}
Der Zusammenhang \textit{magBananen(Schimpanse)} kann als
\begin{itemize}
 \item Prädikat interpretiert werden. Welches den Zustand $W$ einnimmt
 \item 1-Stellige Funktion mit dem Rückgabewert $W$ interpretiert werden.
\end{itemize}
Meiner Ansicht nach sind beide Interpretationen richtig. Wenn ich mich für eine Variante entscheiden müsste, dann würde ich es erneut als \textbf{Prädikat} interpretieren.
\section{6.d)}
\setcounter{equation}{0}
\begin{align*}
\text{equals(5,3)(=F)} && \text{Equals-Zusammenhang}
\end{align*}
Dieses Prädikat drückt den Zusammenhang `Equals` oder `Gleich` aus und nimmt den Zustand $W$ ein, wenn die Beiden Objekte `Gleich` sind oder $F$, wenn die Objekte nicht `Gleich` sind. Hier weitere Beispiele:
\begin{align*}
\text{equals(5,5)(=W)} && \text{Equals-Zusammenhang}\\
\text{equals(1,9)(=F)} && \text{Equals-Zusammenhang}
\end{align*}
\end{document}
