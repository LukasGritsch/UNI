\documentclass[12pt]{article}
\usepackage[utf8]{inputenc}
\usepackage{tabularx}
\usepackage{lmodern,textcomp}
\usepackage{geometry,lipsum}
\usepackage{xcolor}
\usepackage{fancyhdr}
\usepackage{graphicx}
\usepackage{makecell}
\usepackage{eurosym}
\usepackage{amsmath}
\usepackage{pgfplots}
\usepackage[onehalfspacing]{setspace}
% http://ctan.org/pkg/{geometry,lipsum}
\renewcommand{\contentsname}{Inhaltsverzeichnis}
% default
%\geometry{margin=1in}% 1in margin
\geometry{margin=2cm}% 1cm margin

\pagestyle{fancy}
\fancyhf{}
\rhead{\today}
\chead{Lukas Gritsch}
\lhead{eCollaboration}
\rfoot{Seite \thepage}

\begin{document}
\tableofcontents
\pagebreak
\section{Aufgabe Pairprogramming}
Mit wem habe ich an welchem Projekt und wie gearbeitet?
\subsection{Zeitaufwand}
Habe ich dazu mehr oder weniger Zeit benötigt, als hätte ich dies alleine gemacht?
\subsection{Verwendung in der Zukunft}
Werde ich diese Methode in der Zukunft verwenden?
\subsection{Reposetory mit Ergebnis}
Den Link zum MCI-Repo einfügen und eventuell auch gleich den python Code einfügen


\end{document}
