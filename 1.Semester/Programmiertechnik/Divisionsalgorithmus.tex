\documentclass[12pt]{article}
\usepackage[utf8]{inputenc}
\usepackage{geometry,lipsum}
\usepackage{fancyhdr}
\usepackage{sectsty}
\usepackage{multicol}
\usepackage[dvipsnames]{xcolor}
\usepackage{enumitem}
\usepackage{tcolorbox}
\usepackage{csquotes}
\usepackage[backend=biber,style=alphabetic,]{biblatex}
\usepackage{graphicx}
\usepackage{listings}
\usepackage[ngerman]{babel}

\renewcommand{\labelenumii}{\theenumii}
\renewcommand{\theenumii}{\theenumi.\arabic{enumii}.}
\setlength{\leftmarginii}{1.8ex}

\sectionfont{\fontsize{12}{15}\selectfont}
\subsectionfont{\fontsize{12}{15}\selectfont}
\geometry{margin=2cm}
\pagestyle{fancy}
\fancyhf{}
\rhead{\today}
\chead{Lukas Gritsch}
\lhead{Programmiertechnik}
\rfoot{Seite \thepage}
\renewcommand{\figurename}{Abbildung}

\definecolor{mygreen}{rgb}{0,0.6,0}
\definecolor{mygray}{rgb}{0.5,0.5,0.5}
\definecolor{mymauve}{rgb}{0.58,0,0.82}

\lstset{ %
  backgroundcolor=\color{white},   % choose the background color
  basicstyle=\footnotesize,        % size of fonts used for the code
  breaklines=true,                 % automatic line breaking only at whitespace
  captionpos=b,                    % sets the caption-position to bottom
  commentstyle=\color{mygreen},    % comment style
  escapeinside={\%*}{*)},          % if you want to add LaTeX within your code
  keywordstyle=\color{blue},       % keyword style
  stringstyle=\color{mymauve},     % string literal style
}

\begin{document}
\section{Natürliche Sprache}
\begin{enumerate}
 \item Ein User gibt ein ganze Zahl über die Konsole ein.
    \stepcounter{enumi}
 \begin{enumerate}
      \item Diese Eingabe wird überprüft, ob diese auch eine ganze Zahl ist. Sollte dies nicht der Fall sein, wird eine Fehlermeldung ausgegeben und die Eingabe wiederholt.
      \item Ist die Eingabe in Ordnung, wird dies der Variable \textbf{a} zugewiesen.
 \end{enumerate}
 \item Ein User gibt eine zweite ganze Zahl ein, welche nicht 0 sein darf.
     \stepcounter{enumi}
 \begin{enumerate}
 \item Diese Eingabe wird überprüft, ob diese auch eine ganze Zahl ist und nicht 0 ist. Sollte dies nicht der Fall sein, wird eine Fehlermeldung ausgegeben und die Eingabe wiederholt.
 \item Ist die Eingabe in Ordnung, wird dies der Variable \textbf{b} zugewiesen.
\end{enumerate}
 \item Jetzt wird die Berechnung $\frac{a}{b}$ durchgeführt und der Variable \textbf{c} zugewiesen.
 \item \textbf{c} wird nun mittels \textbf{print()} in der Konsole wieder ausgegeben.
\end{enumerate}
\pagebreak
\section{Graphische Repräsentation}
\begin{figure}[h]
 \begin{center}
  \includegraphics[scale=0.43]{DivMitRest}
 \end{center}
 \caption{Flussdiagramm/Programmablaufsplan Division mit Rest}
\end{figure}
\pagebreak
\section*{(Optional) Als SourceCode}
\begin{lstlisting}[caption={Umsetzung des obigen Flussdiagramm/Programmablaufsplan in Python},language=Python,frame=single, numbers=left, breaklines=true]
def getVarA():
    print("Bitte geben Sie ein ganze Zahl ein.")
    hEin = input()
    try:
        a = int(hEin)
        return a
    except:
        print("Eingabe muss ein ganze Zahl sein!")
        return getVarA()

def getVarB():
    b = getVarA()
    if b != 0:
        return b
    else:
        print("Eingabe muss ein ganze Zahl sein und darf nicht 0 sein!")
        return getVarB()

def divMitrest():
    a = getVarA()
    b = getVarB()

    c = a/b

    print(c)

divMitrest()
\end{lstlisting}
\end{document}a
